\documentclass[10pt,a4paper]{article}
\usepackage[utf8]{inputenc}
\usepackage{amsmath}
\usepackage{amsfonts}
\usepackage{amssymb}
\usepackage{color}
\begin{document}

\section{Espacios verticales}
La primera línea separada un espacio pequeño 
\smallskip \\
La línea siguiente \\
La segunda línea separada un espacio medio 
\medskip \\
La línea siguiente \\
La tercera línea separada por un espacio grande 
\bigskip \\
La línea siguiente \\
La cuarta línea separada por un espacio grande 
\vspace{3cm} \\
La línea siguiente 

\section{Espacios horizontales}
Una palabra se separa 2.54cm \hspace{2.54cm} de la siguiente \\
La siguiente frase se 
\hfill{rellena de blancos hasta el final}

\section{Párrafos}
El texto centrado:
\begin{center}
  La primera línea del parrafo 
  \vspace{2cm} \\
  La segunda línea del parrafo 
  \vspace{0.5cm} \\
  La tercera línea del parrafo \\ 

\end{center}
 El texto a la derecha:
\begin{flushright}
  La primera línea del parrafo \\
  La segunda línea del parrafo \\
  La tercera línea del parrafo \\ 
\end{flushright}

 El texto a la izquierda:
\begin{flushleft}
  La primera línea del parrafo \\
  La segunda línea del parrafo \\
  La tercera línea del parrafo \\ 
\end{flushleft}

El texto con citas:
\begin{quote}
  - Hola - dijo Pepe \\
  - Adiós - dijo Hulio \\ 
\end{quote}

\section{Tamaños de letra}
Los distintos tamaños son: \\

{\tiny Esto es tamaño tiny.} \\
{\scriptsize Esto es tamaño scriptsize.} \\
{\footnotesize Esto es tamaño footnotesize.} \\
{\small Esto es tamaño small.} \\
{\normalsize Esto es tamaño normalsize.} \\
{\large Esto es tamaño large.} \\
{\Large Esto es tamaño Large.} \\
{\LARGE Esto es tamaño LARGE.} \\

\section{Colecciones}
Los dias de la semana son:
\begin{enumerate}
\item lunes
\item martes
\item miercoles
\item jueves
\item viernes
	\begin{enumerate}
	\item 24 de noviembre
		\begin{enumerate}
		\item viernes negro
			\begin{enumerate}
			\item uno
			\item dos
			\end{enumerate}
		\item viernes blanco
		\end{enumerate}
	\item black friday
	\end{enumerate}
\item sabado
\item domingo
\end{enumerate}

\section{Colecciones sin numerar}
\begin{itemize}
\item lunes
\item martes
\item miercoles
\item jueves
\item viernes
	\begin{itemize}
	\item ok
	\item ko
	\end{itemize}
\item sabado
\item domingo	
\end{itemize}

\section{Colecciones con y sin numerar}
\begin{enumerate}
\item lunes
\item martes
\item miercoles
	\begin{itemize}
	\item miercoles blanco
		\begin{enumerate}
        \item jijijijijijijijijijiji		
		\end{enumerate}
	\item miercoles azul	
	\end{itemize}
\item jueves
\item viernes
\item sabado
\item domingo	
\end{enumerate}
\section{Colores con nombre}
Probamos los siguientes colores
\begin{itemize}
\item \textcolor{green}{Verde}
\item \textcolor{yellow}{Amarillo}
\item \textcolor{blue}{Azul}
\item \textcolor{red}{Rojo}
\item \textcolor{magenta}{Magenta}
\item \textcolor{cyan}{Cyan}
\end{itemize}
Probamos los siguentes colores en rgb :V
\begin{itemize}
\item \textcolor[rgb]{0.25,1,0.78}{ni idea de que color}
\item \textcolor[rgb]{0.32,0.145,0.200}{ni idea hulio} 
\item \textcolor[rgb]{0.122,0.55,1}{hola world}
\item \textcolor[rgb]{1,0,0}{Viva el bicho}
\item \textcolor[rgb]{0,1,1}{cyan}
\item \textcolor[rgb]{1,1,0}{yellow}
\item \textcolor[rgb]{0,1,0}{green}
\item \textcolor[rgb]{0,0,1}{blue}
\item \textcolor[rgb]{1,0,1}{magenta}
\end{itemize}
\section{Colores con cmyk}
\begin{itemize}
\item \textcolor[cmyk]{0,1,1,0}{Rojo}
\item \textcolor[cmyk]{1,0,1,0}{Verde}
\item \textcolor[cmyk]{1,1,0,0}{Azul}
\item \textcolor[cmyk]{0,0,1,0}{Amarillo}
\item \textcolor[cmyk]{0,1,0,0}{Magenta}
\item \textcolor[cmyk]{1,0,0,0}{Cyan}
\end{itemize}
\section{Modo matemático basico}
Una ecuación de segundo grado es
$ax^2+bx+c=0$ en la misma línea \\
Una ecuación de segundo grado en otra línea 
$$ax^2+bx+c=0$$ En otra línea \\
Una sucesión de números se compone de:
$$ x_n = x_1 + x_2 \ldots + x_n $$ 
\end{document}